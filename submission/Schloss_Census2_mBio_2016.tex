\documentclass[11pt,]{article}
\usepackage{lmodern}
\usepackage{amssymb,amsmath}
\usepackage{ifxetex,ifluatex}
\usepackage{fixltx2e} % provides \textsubscript
\ifnum 0\ifxetex 1\fi\ifluatex 1\fi=0 % if pdftex
  \usepackage[T1]{fontenc}
  \usepackage[utf8]{inputenc}
\else % if luatex or xelatex
  \ifxetex
    \usepackage{mathspec}
    \usepackage{xltxtra,xunicode}
  \else
    \usepackage{fontspec}
  \fi
  \defaultfontfeatures{Mapping=tex-text,Scale=MatchLowercase}
  \newcommand{\euro}{€}
\fi
% use upquote if available, for straight quotes in verbatim environments
\IfFileExists{upquote.sty}{\usepackage{upquote}}{}
% use microtype if available
\IfFileExists{microtype.sty}{%
\usepackage{microtype}
\UseMicrotypeSet[protrusion]{basicmath} % disable protrusion for tt fonts
}{}
\usepackage[margin=1.0in]{geometry}
\ifxetex
  \usepackage[setpagesize=false, % page size defined by xetex
              unicode=false, % unicode breaks when used with xetex
              xetex]{hyperref}
\else
  \usepackage[unicode=true]{hyperref}
\fi
\hypersetup{breaklinks=true,
            bookmarks=true,
            pdfauthor={},
            pdftitle={The status of the archaeal and bacterial census: an update},
            colorlinks=true,
            citecolor=blue,
            urlcolor=blue,
            linkcolor=magenta,
            pdfborder={0 0 0}}
\urlstyle{same}  % don't use monospace font for urls
\setlength{\parindent}{0pt}
\setlength{\parskip}{6pt plus 2pt minus 1pt}
\setlength{\emergencystretch}{3em}  % prevent overfull lines
\providecommand{\tightlist}{%
  \setlength{\itemsep}{0pt}\setlength{\parskip}{0pt}}
\setcounter{secnumdepth}{0}

%%% Use protect on footnotes to avoid problems with footnotes in titles
\let\rmarkdownfootnote\footnote%
\def\footnote{\protect\rmarkdownfootnote}

%%% Change title format to be more compact
\usepackage{titling}

% Create subtitle command for use in maketitle
\newcommand{\subtitle}[1]{
  \posttitle{
    \begin{center}\large#1\end{center}
    }
}

\setlength{\droptitle}{-2em}
  \title{\textbf{The status of the archaeal and bacterial census: an update}}
  \pretitle{\vspace{\droptitle}\centering\huge}
  \posttitle{\par}
  \author{}
  \preauthor{}\postauthor{}
  \date{}
  \predate{}\postdate{}

\usepackage{helvet} % Helvetica font
\renewcommand*\familydefault{\sfdefault} % Use the sans serif version of the font
\usepackage[T1]{fontenc}

\usepackage[none]{hyphenat}

\usepackage{setspace}
\doublespacing
\setlength{\parskip}{1em}

\usepackage{lineno}

\usepackage{pdfpages}

% Redefines (sub)paragraphs to behave more like sections
\ifx\paragraph\undefined\else
\let\oldparagraph\paragraph
\renewcommand{\paragraph}[1]{\oldparagraph{#1}\mbox{}}
\fi
\ifx\subparagraph\undefined\else
\let\oldsubparagraph\subparagraph
\renewcommand{\subparagraph}[1]{\oldsubparagraph{#1}\mbox{}}
\fi

\begin{document}
\maketitle

\vspace{35mm}

Running title: The archaeal and bacterial census

\vspace{35mm}

Patrick D. Schloss\textsuperscript{1\(\dagger\)}, Rene
Girard\textsuperscript{2}, Thomas Martin\textsuperscript{2}, Joshua
Edwards\textsuperscript{2}, and J. Cameron
Thrash\textsuperscript{2\(\dagger\)}

\vspace{40mm}

\(\dagger\) To whom correspondence should be addressed:
\href{mailto:pschloss@umich.edu}{\nolinkurl{pschloss@umich.edu}} and
\href{mailto:thrashc@lsu.edu}{\nolinkurl{thrashc@lsu.edu}}

1. Department of Microbiology and Immunology, University of Michigan,
Ann Arbor, MI 48109

2. Department of Biological Sciences, Louisiana State University, Baton
Rouge, LA 70803

\newpage

\linenumbers

\subsection{Abstract}\label{abstract}

A census is typically carried out for people across a range of
geographical levels; however, microbial ecologists have implemented a
molecular census of bacteria and archaea by sequencing their 16S rRNA
genes. We assessed how well the census of full-length 16S rRNA gene
sequences is proceeding in the context of recent advances in high
throughput sequencing technologies because full-length sequences are
typically used as references for classification of the short sequences
generated by newer technologies. Among the 1,411,234 and 53,546
full-length bacterial and archaeal sequences sequences, 94.5\% and
95.1\% of the bacterial and archeaeal sequences, respectively, belonged
to operational taxonomic units (OTUs) that have been observed more than
once. Although these metrics suggest that the census is approaching
completion, 29.2\% of the bacterial and 38.5\% of the archaeal OTUs have
been observed more than once. Thus, there is still considerable
diversity to be explored. Unfortunately, the rate of new full-length
sequences has been declining and new sequences are primarily being
deposited by a small number of studies. Furthermore, sequences from soil
and aquatic environments, which are known to be rich in bacterial
diversity, only represent 7.8 and 16.5\% of the census while sequences
associated with host-associated environments represent 55.0\% of the
census. Continued use of traditional approaches and new technologies
such as single cell genomics and short read assembly are likely to
improve our ability to sample rare OTUs if it is possible to overcome
this sampling bias. The success of ongoing efforts to use short read
sequencing to characterize archaeal and bacterial communities requires
that researchers strive to expand the depth and breadth of this census.

\subsubsection{Importance}\label{importance}

The biodiversity contained within the bacterial and archaeal domains
dwarfs that of the eukaryotes and the services these organisms provide
to the biosphere are critical. Surprisingly, we have done a relatively
poor job of formally tracking the quality of the biodiversity as
represented in full-length 16S rRNA genes. By understanding how this
census is proceeding, it is possible to suggest the best allocation of
resources for advancing the census. We found that the ongoing effort has
done an excellent job of sampling the most abundant organisms, but
struggles to sample the more rare organisms. Through the use of new
sequencing technologies we should be able to obtain full-length
sequences from these rare organisms. Furthermore, we suggest that by
allocating more resources to sampling environments known to have the
greatest biodiversity we will be able to make significant advances in
our characterization of archaeal and bacterial diversity.

\newpage

\subsection{Introduction}\label{introduction}

The effort to quantify the number of different organisms in a system
remains fundamental to understanding ecology (1, 2). At the scale of
microorganisms, small physical sizes, morphological ambiguity, and
highly variable population sizes complicate this process. Furthermore,
creating standards for delimiting what makes one microbe ``different''
from another has been contentious (3, 4). In spite of these challenges,
we continue to peel back the curtain on the microbial world with the aid
of more and more informative, if still limited, technologies like
cultivation, 16S rRNA gene surveys, single cell technologies, and
metagenomics.

Generating a comprehensive understanding of any system with a single
gene may seem a fool's errand, yet we have learned a considerable amount
regarding the diversity, dynamics, and natural history of microorganisms
using the venerable 16S rRNA gene. In 1983, the full-length 16S rRNA
gene sequence of \emph{Escherichia coli} (accession J01695) was released
as part of NCBI's GenBank making it the first of what is now more than
10 million 16S rRNA gene sequences to be deposited into the database
(5). 16S rRNA gene accessions represent nearly one-third of all
sequences deposited in GenBank, making it the best-represented gene. As
Sanger sequencing has given way to so-called ``next generation
sequencing'' technologies, hundreds of millions of 16S rRNA gene
sequences have been deposited into the NCBI's Sequence Read Archive. The
expansion in sequencing throughput and increased access to sequencing
technology has allowed for more environments to be sequenced at a deeper
coverage, resulting in the identification of novel taxa. The ability to
obtain sequence data from microorganisms without cultivation has
radically altered our perspective of their role in nearly every
environment from deep ocean sediment cores (e.g.~accession AY436526) to
the International Space Station (e.g.~accession DQ497748).

Previously, Schloss and Handelsman (6) assigned the 56,215 partial 16S
rRNA gene sequences that were available in the Ribosomal Database
Project to operational taxonomic units (OTUs) and concluded that the
sampling methods of the time were insufficient to identify the
previously estimated 10\textsuperscript{7} to 10\textsuperscript{9}
different species (7, 8). That census called for a broader and deeper
characterization of all environments. Refreshingly, this challenge was
largely met. There have been major investments in studying the Earth's
microbiome using 16S rRNA gene sequencing through initiatives such as
the Human Microbiome Project (9), the Earth Microbiome Project (10), and
the International Census of Marine Microorganisms (11). But most
importantly, the original census was performed on the cusp of radical
developments in sequencing technologies. That advancement has moved the
generation of sequencing throughput from large sequencing centers to
individual investigators and leveraged their diverse interests to expand
the representation of organisms and environments represented in public
databases.

It is disconcerting that the increase in sequencing volume has come at
the cost of sequence length. The commonly used MiSeq-based sequencing
platform from Illumina is extensively used to sequence the approximately
250 bp V4 hypervariable region of the 16S rRNA gene; other schemes have
used different parts of the gene that are generally shorter than 500 bp.
The number of OTUs that are inferred from the sequencing data when using
different regions within the 16S rRNA gene can vary considerably and the
genetic diversity within these regions typically has only a modest
correlation with the genetic diversity of the full-length sequence (12,
13). Thus, it remains unclear to what degree richness estimates from
short read technologies over or underestimate the numbers from
full-length sequences. Furthermore, we likely lack the full-length
reference sequences necessary to adequately classify the novel
biodiversity we are sampling when we generate 100-times the sequence
data from a community than we did using full-length sequencing.

Here we update the status of the archaeal and bacterial census with
full-length 16S rRNA gene sequences. In the 13 years since the
collection of data for Schloss and Handelsman's initial census, the
number of full-length sequences has grown exponentially, despite the
overwhelming contemporary focus by most researchers on short-read
technologies. This update to the census allows us to evaluate the
relative sampling thoroughness for different environments and clades and
make an argument for the continued need to collection full-length
sequence data from many systems that have a long history of study. As
researchers consider coalescing into a Unified Microbiome Initiative
(14), it will be important to balance the need for mechanism-based
studies with the need to generate full-length reference sequences from a
diversity of environments.

\subsection{Results and Discussion}\label{results-and-discussion}

\textbf{\emph{The status of the bacterial and archaeal census.}} To
assess the field's progress in characterizing the biodiversity of
bacteria and archaea, we assigned each 16S rRNA gene sequence to OTUs
using distance thresholds that varied between 0 and 20\%. Although it is
not possible to link a specific taxonomic level (e.g.~species, genus,
family, etc.) to a specific distance threshold, we selected distances of
0, 3, 5, 10, and 20\% because they are widely regarded as representing
the range of genetic diversity of the 16S rRNA gene within each domain.
By rarefaction, it was clear that the ongoing sampling efforts have
started to saturate the number of current OTUs. After sampling 1,411,234
near full-length bacterial 16S rRNA gene sequences we have identified
217,645, 108,950, 66,819, 15,743, and 3,731 OTUs at the respective
thresholds (Figure 1A, Table 1). Using only the OTUs generated using a
3\% threshold, we calculated a 94.5\% Good's coverage (percent of
sequences belonging to OTUs that have been observed more than once), but
only 29.2\% OTU coverage (percent of the OTUs that have been observed
more than once). Paralleling the bacterial results, after sampling
53,546 archaeal 16S rRNA gene sequences we have identified 11,040,
4,252, 2,364, 812, and 110 OTUs (Figure 1B, Table 1). Using only the
OTUs generated with a 3\% threshold, we calculated a 95.1\% Good's
coverage, but only 38.5\% OTU coverage. These results indicate that
regardless of the domain, continued sampling with the current strategies
for generating full-length sequences will largely reveal OTUs that have
already been observed, even though a large fraction of OTUs have only
been sampled once. Considering more than 70.8\% of the OTUs have only
been observed once, it is likely that an even larger number of OTUs have
yet to be sampled for both domains.

\textbf{\emph{Sequencing efforts are a source of bias in the census.}}
One explanation for the large number of OTUs that have only been
observed once is that with the the broad adoption of sequencing
platforms that generate short sequence reads, the rate of full-length
sequence generation has declined. In fact, since 2009 the number of new
bacterial sequences generated has slowed to an average of 189,960
sequences per year (Figure 2A). Although this is still an impressive
number of sequences, since 2007 the number of new bacterial OTUs has
plateaued at an average of 11,184 new OTUs per year (Figure 2B). Given
the expense of generating full-length sequences using the Sanger
sequencing technology and the transition to other platforms at that
time, we expected that the large number of sequences were being
deposited by a handful of large projects. Indeed, when we counted the
number of submissions responsible for depositing 50\% of the sequences,
we found that with the exception of 2006 and 2013, eight or fewer
studies were responsible for depositing the majority of the full-length
sequences each year since 2005 (Figure 2C). Between 2009 and 2012,
908,190 total sequences were submitted and 6 submissions from 5 studies
were responsible for depositing 550,274 (60.6\% of all sequences). These
studies generated sequences from the human gastrointestinal tract (15),
human skin (16, 17), murine skin (18), and hypersaline microbial mats
(19). The heavy focus on host-associated communities is reflected in the
rarefaction curve for this category (Figure 1C). In contrast to recent
years, between 1995 and 2006, an average of 39.3 studies were
responsible for submitting more than half of the sequences each year.
Although the recent deep surveys represent significant contributions to
our knowledge of bacterial biogeography, their small number and lack of
environmental diversity is indicative of the broader problems in
advancing the bacterial census.

The depth of sequencing being done to advance the archaeal census has
been 26-times less than that of the bacterial census (Table 1). The
annual number of sequences submitted has largely paralleled that of the
bacterial census with a plateau starting in 2009 and an average of 7,075
sequences each year since then. The number of new archaeal OTUs
represented by these sequences began to slow in 2005 with an average of
355.5 new OTUs per year. With the exception of 2012 and 2014, the number
of submissions responsible for more than 50\% of the archaeal sequences
submitted per year has varied between 2 and 11 submissions per year. The
clear bias towards sequencing bacterial 16S rRNA genes has limited the
ability to more fully characterize the biodiversity of the archaea,
which is clearly reflected in the relatively meager sampling effort
across habitats, compared to bacteria (Figure 1D),

\textbf{\emph{The ability to sample archaea and bacteria is
taxonomically skewed.}} The Firmicutes, Proteobacteria, Actinobacteria,
and Bacteroidetes represent 89.2\% of the bacterial sequences and the
Euryarchaeota and Thaumarchaeota 86.5\% of the archaeal sequences. We
sought to understand how the representation of individual phyla has
changed relative to the state of the census in 2006. We used 2006 as a
reference point for calibrating the dynamics of the bacterial and
archaeal censuses since that was the year that the first highly
parallelized 16S rRNA gene sequence dataset was published (20). Based on
the representation of sequences within the SILVA database, in 2006 there
were 61 bacterial and 18 archaeal phyla. Since then there have been 4
new bacterial (CKC4, OC31, S2R-29, and SBYG-2791) and 2 new archaeal
candidate phyla (Ancient Archaeal Group and TVG8AR30). Relative to the
overall sequencing trends before and after 2006, several phyla stand out
for being over and underrepresented in sequence submissions (Figure 3).
Among the bacterial phyla with at least 1,000 sequences, Atribacteria
and Kazan-3B-09 were sequenced 4-fold more often while
Deinococcus-Thermus and Tenericutes were sequenced 2-fold less often
than would have been expected since 2006. Among the archaeal phyla with
at least 1,000 sequences, the Thaumarchaeota were sequenced 2.0-fold
more often and the Crenarchaeota were sequenced 6.7-fold less often than
expected. Together, these results demonstrate a change in the
phylum-level lineages represented in the census from before and after
2006 and encouragingly, show that some underrepresented phyla are
becoming better sampled.

\textbf{\emph{Focusing the census by environment.}} We were able to
assign 89.3 and 95.1\% of the sequences to one of seven broad
environmental categories based on the metadata that accompanied the
SILVA database (Tables 1). Across these broad categories there was wide
variation in the number of sequences that have been sampled. Among
bacterial sequences, the three best represented groups were from
host-associated (N=804,585), aquatic (N=214,085), and built environment
(N=108,799) sources. Among the archaeal sequences the three best
represented groups were the same, but ordered differently: aquatic
(N=34,400), built environment (N=7,286), and host-associated (N=5,597)
(Figure 1C,D)). For both domains, soil samples were the fourth most
represented category (bacteria: 74,870; archaea: 2,517). The orders of
these categories was surprising considering soil and aquatic
environments harbor the most microbial biomass and biodiversity (21). In
spite of wide variation in sequencing depth and coverage (Table 1), the
interquartile range across the fine-level categories for the bacterial
OTU coverage only varied between 34.5 to 40.0 (median coverage=36.7\%).
The interquartile range in the OTU coverage by environment for the
archaeal data was 41.5 to 53.1 (median coverage=44.9\%). The archaeal
coverage was higher than that of the bacterial OTU coverage for all
categories except the food-associated, plant surface, and other
invertebrate categories. Across all categories, the bacterial and
archaeal sequencing data represented a limited number of phyla (Figure
4). Among the bacterial data, the fine-scale categories were dominated
by Proteobacteria (N=24), Firmicutes (N=2), and Actinobacteria (N=1) and
among the archaeal data, they were dominated by Euryarchaeota (N=16),
Thaumarchaeota (N=10), and Aenigmarchaeota (N=1). Regardless, there were
clear phylum-level signatures that differentiated the various
categories. Within each of the bacterial and archaeal phyla, there was
considerable variation in the relative abundance of each across the
categories confirming that taxonomic signatures exist to differentiate
different environments even at a broad taxonomic level.

\textbf{\emph{The cultured census.}} In the 2004 bacterial census, there
was concern expressed that although culture-independent methods were
significantly enhancing our knowledge of microbial life, there were
numerous bacterial phyla with no or only a few cultured representatives.
To update this assessment, we identified those sequences that came from
cultured and uncultured organisms. Overall, 18.9\% of bacterial
sequences and 6.8\% of archaeal sequences have come from isolated
organisms. Comparing the fraction of sequences deposited during and
before 2006 from isolates to those collected after 2006, we found that
culturing rates lag by 2.4 and 2.5-fold for bacteria and archaea,
respectively. Among the 65 bacterial phyla, 24 have no cultured
representatives and 14 of the 20 archaeal phyla have no cultured
representatives. This lag is likely due to the differences in throughput
of culture-dependent and -independent approaches. Of the phyla with at
least one cultured representative, the median percentage of sequences
coming from a culture was only 2.8\% for the bacterial phyla and 1.7\%
for the archaeal phyla (Figure 5). Even though many phyla have cultured
representatives, there is still a skew in the representation of most
phyla found in cultivation efforts.

Considering the possibility that large culture-independent sequencing
efforts may only be re-sequencing organisms that already exist in
culture, we asked what percentage of OTUs had at least one cultured
representative. We found that 16.9\% of the 117,385 bacterial OTUs and
13.1\% of the 4,574 archeael OTUs had at least one cultured
representative (Figure 5). Comparing the percentage of sequences with
cultured representatives to the percentage of OTUs containing a sequence
from a cultured representative revealed a strong cultivation bias within
the Firmicutes, which had a higher percentage of sequences generated by
cultivated representatives than would be expected based on the number of
cultured organisms represented by OTUs (Figure 5). This likely reflects
the extremely high number of cultivated biomedically relevant cultivars
from genera such as \emph{Bacillus}, \emph{Streptococcus},
\emph{Lactobacillus}, \emph{Staphylococcus}, and others. Conversely,
many phyla, including Cyanobacteria, Actinobacteria, Bacteroidetes, and
Nitrospirae, had a lower percentage of sequences belonging to cultivated
representatives than would be expected based on the percentage of OTUs
that have sequences from cultured organisms, indicating that the
cultivation efforts in these clades are relatively inefficient with
regards to available diversity. Nevertheless, it is clear that the
majority of OTUs from any phylum remain uncultivated, to say nothing of
the diversity of organisms that may be encapsulated within the 97\%
sequence identity cutoff.

\textbf{\emph{New technologies to access novel biodiversity.}} Given the
shift from Sanger sequencing to platforms that offer higher throughput
but shorter reads, we are concerned that our ability to harvest
full-length sequences from communities will remain stalled. Several
culture-independent methods have been developed that offer the ability
to obtain full-length sequences of the 16S rRNA gene and even the
complete genome. These have included single cell genomics (22) and
assembly of short 16S rRNA gene fragments using data generated from PCR
amplicons or metagenomic shotgun sequence data with the
Expectation-Maximization Iterative Reconstruction of Genes from the
Environment (EMIRGE) algorithm (23, 24). To test the ability of these
technologies to expand our knowledge of microbial diversity beyond that
of traditional approaches, we compared the overlap of OTUs found using
each of the new methods with the traditional approaches (Figure 6).
Utilizing the 16S rRNA gene sequences extracted from the single-cell
genomes available on the Integrated Microbial Genomes (IMG) system (25),
we identified 311 bacterial and 70 archaeal sequences, which were
assigned to 115 and 27 bacterial and archaeal OTUs, respectively.
Interestingly, only 8.7 and 3.7\% of the bacterial and archaeal single
celled OTUs, respectively, had not been observed by previous efforts.
Next, we identified six studies that utilized EMIRGE to assemble 16S
rRNA gene sequences from metagenomic sequences (23, 26--30). Together
these studies assembled 599 bacterial and 9 archaeal full-length
sequences, which were assigned to 335 and 7 bacterial and archaeal OTUs,
respectively. Only 40.6 and 60.3\% of the bacterial OTUs generated by
this approach were previously identified by this traditional cultivation
and PCR-based approaches, respectively. Although the application of this
approach to Archaea has been limited, it was still surprising that 85.7
and 85.7\% of the archaeal OTUs had been previously recovered by
traditional cultivation and PCR-based approaches, respectively. Finally,
we pooled 76,080 bacterial sequences from five studies that utilized
EMIRGE to assemble 16S rRNA gene sequences from fragmented amplicons
(24, 31--34). These sequences were assigned to 40,213 OTUs. We were
surprised that only 7.6\% of these OTUs were previously found by a more
traditional approach. Although these PCR-based EMIRGE results may be
valid, the high degree of novelty that was observed suggests that the
error of the assembled reads may be too high for generating reference
sequences. Each of these methods represent promising opportunities to
continue the bacterial census using full-length sequences as well as
genomic information.

\subsection{Conclusions}\label{conclusions}

It is clear that considerable biodiversity has been discovered since the
first census in 2004. However, much of it has been biased towards
particular phyla and environments. Our analysis suggests that 94.5\% of
new full-length bacterial and archaeal sequences are likely to have
already been seen. Meanwhile, 29.2\% of bacterial and 38.5\% of archaeal
OTUs have only been observed once. In spite of current estimates
suggesting the global bacterial species richness may be as high as
10\textsuperscript{12} species (35), the current census based on
full-length 16S rRNA gene sequences suggests that existing sampling
methods will prevent us from acquiring full-length sequences for that
level of diversity. As we have shown, current strategies repeatedly
sample the same OTUs and do a poor job of resampling rarer populations.
Given this low level of OTU coverage, it is likely that there are many
more bacterial and archaeal populations yet to be sampled.

There are several additional reasons to suspect that the current census
should be considered conservative. First, we found that most sequences
recently deposited into public databases are being made by a small
number of projects that have deeply sampled similar environments, and
the number of full-length reads deposited into the databases has
stalled. Second, it is widely acknowledged that 16S rRNA gene primers
are biased; these biases are amplified when designing primers to amplify
subregions used in sequencing short reads (36). Assembly of metagenomic
data has shown the presence of introns in the 16S rRNA genes of
organisms within the so-called ``Candidate Phyla Radiation''
(e.g.~Saccharibacteria (TM7), Peregrinibacteria, Berkelbacteria (ACD58),
WWE3 Microgenomates (OP11), Parcubacteria (OD1), et al.) that would
preclude detection with standard PCR-based approaches (37, 38). Third,
the willingness of researchers to contribute their sequences and the
metadata describing the environment that the sequences were sampled from
is critical for assessing the progress of the census and to accrue the
benefits from having full-length sequences in the databases. As an
illustration of this problem, only 5 of the 11 studies that used the
EMIRGE algorithm deposited their sequences in GenBank. This makes the
sequences from the other studies effectively invisible to the search
algorithms used by 16S rRNA gene-specific databases to harvest
sequences. As assembly and long read technologies advance, a mechanism
is needed to assess the quality of the consensus sequences and to make
them easily accessible to the 16S rRNA gene-specific databases.

Efforts to census archaea and bacteria using short read technology such
as the International Census of Marine Microbes, the Earth Microbiome
Project, and the Human Microbiome Project have significantly advanced
our knowledge of archaeal and bacterial biogeography; however, these
analyses have demonstrated the limitations of databases and taxonomies
that are based on sequences from common and abundant organisms. During
the period prior to the introduction of massively parallelized high
throughput sequencing, it was common for a study to generate dozens or
hundreds of sequences per sample. The existing databases that are used
for classifying sequences are based on those sequences, which represent
organisms that are generally abundant. We hypothesize that recent
difficulties obtaining adequate classification for short sequences
captured from more rare organisms are because our databases do not
contain full-length references for those sequences. We fear that these
trends will worsen unless researchers can leverage new sequencing and
cultivation technologies to generate large numbers of full-length
sequences from a large number of diverse samples.

Novel technologies such as single-cell genomics, metagenomics, and
algorithms to recover full-length sequences from new sequencing
platforms have demonstrated promise in circumventing previous
limitations in identifying new OTUs. Using EMIRGE to assemble fragmented
16S rRNA gene amplicons may allow us to obtain deep coverage of
communities; however, it is still unclear how faithful the assembled
sequence is to that of the original organism. Additional sequencing
technologies also offer the ability to directly generate full-length
sequences, such as PacBio and potentially Oxford Nanopore. Initial
application of PacBio to sequencing full-length fragments suggests that
the sequences suffered from a high error rate (39). To obtain a more
direct investigation of rare organisms, microbiologists are developing
novel cultivation and single cell genomics techniques ({\textbf{???}},
40--42). The ability to enrich or select for specific populations using
these approaches could limit the need for redundant brute force
sequencing. These approaches are still in active development, and we
hope that through continuous refinement, they may allow us to
significantly improve the coverage of OTUs in public databases.

\subsection{Materials and Methods}\label{materials-and-methods}

\textbf{\emph{Sequence data curation.}} The July 19, 2015 release of the
ARB-formatted SILVA small subunit (SSU) reference database (SSU Ref
v.123) was downloaded from
\url{http://www.arb-silva.de/fileadmin/silva_databases/release_123/ARB_files/SSURef_123_SILVA_19_07_15_opt.arb.tgz}
(43). This release is based on the EMBL-EBI/ENA Release 123, which was
released in March 2015. The SILVA curators identify potential SSU
sequences using keyword searches and sequence-based search using RNAmmer
(\url{http://www.arb-silva.de/documentation/release-123/}). The SILVA
curators then screened the 7,168,241 resulting sequences based on a
minimum length criteria (\textless{}300 nt), number of ambiguous base
calls (\textgreater{}2\%), length of sequence homopolymers
(\textgreater{}2\%), presence of vector contamination
(\textgreater{}2\%), low alignment quality value (\textless{}75), and
likelihood of being chimeric (Pintail value \textless{} 50). Of the
remaining sequences, the bacterial reference set retained those
bacterial sequences longer than 1,200 nucleotides and the archaeal
reference set retained those archaeal sequences longer than 900
nucleotides. The aligned 1,515,024 bacterial and 59,240 archaeal
sequences were exported from the database using ARB along with the
complete set of metadata. Additional sequence data was included from
single-cell genomes available on the Integrated Microbial Genomes (IMG)
system (25), many of which were recently obtained via the GEBA-MDM
effort in Rinke et al. (22). ``SCGC'' was searched on the IMG database
March 12, 2015 to download the bacterial (N=249) and archaeal (N=46) 16S
rRNA gene sequences and their associated metadata. Further, sequences
generated from amplicon and shotgun metagenomic data using the EMIRGE
program were also included (23, 24). The IMG and EMIRGE sequences were
aligned against the respective SILVA-based reference using mothur (44).
The aligned bacterial and archaeal sequence sets were pooled and
processed in parallel. Using mothur, sequences were further screened to
remove any sequence with more than 2 ambiguous base calls and trimmed to
overlap the same alignment coordinates. The sequences in the resulting
bacterial dataset overlapped bases 113 through 1350 of an \emph{E. coli}
reference sequence (V00348) and had a median length of 1,233 nt. The
sequences in the resulting archaeal dataset overlapped positions 362 to
937 of a \emph{Sulfolobus solfataricus} reference sequence (X03235) and
had a median length of 580 nt. The archaeal sequences were considerably
shorter than their initial length because it was necessary to find a
common overlapping region across the sequences. The final datasets
contained 1,411,234 bacterial and 53,546 archaeal 16S rRNA gene
sequences. Sequences were assigned to OTUs using the average neighbor
clustering algorithm (45).

\textbf{\emph{Metadata curation.}} The metadata that was contained
within the SSU Ref database was used to expand our analysis beyond a
basic count of sequences and the number of OTUs in each domain. The
environmental origins of the 16S rRNA gene sequences were manually
classified using seven broad ``coarse'' categories, and further refined
to facilitate additional analyses with twenty-six more specific ``fine''
categories (Table S1). These were assigned based on manual curation of
the ``isolation\_source'' category within the ARB database associated
with each of the sequences. For source definitions that were not
identifiable by online searches, educated guesses were made or they were
placed into the coarse ``Other'' category. There were 151,669 bacterial
and 2,565 archaeal sequences where an ``isolation\_source'' term was not
collected. We ascertained whether a sequence came from a cultured
organism by including those sequences that had data in their ``strain''
or ``isolate'' fields within the database and excluded any sequences
that had ``Unc'' as part of their database name as this is a convention
in the database that represents sequences from uncultured organisms.
Complete tables containing the ARB-provided metadata, taxonomic
information, OTU assignment, and our environmental categorizations are
available at FigShare for the bacterial
(\url{https://dx.doi.org/10.6084/m9.figshare.2064927}) and archaeal
(\url{https://dx.doi.org/10.6084/m9.figshare.2064942}) data.

\textbf{\emph{Calculating coverage.}} Sequencing coverage
(C\textsubscript{Sequence}) was quantified by two methods. The first was
to use Good's coverage according to

\[C_{Sequence} = 1 - \frac{n_1}{N_t}\]

where n\textsubscript{1} is the number of OTUs represented by only one
sequence and N\textsubscript{t} is the total number of sequences (46).
Although Good's coverage provides information about the success of the
sequencing effort in sampling the most abundant organisms in a
community, it does not directly provide information about the success of
the sequencing effort in recovering previously unobserved OTUs. To
quantify the ability of sequencing to identifying novel OTUs or, in
other words, to quantify the ``distance'' in the peak of the rarefaction
curves to their hypothetical asymptote, we defined ``OTU coverage''
(C\textsubscript{OTU}) as

\[C_{OTU} = 1 - \frac{n_1}{S_t}\]

where S\textsubscript{t} is the total number of OTUs. Whereas Good's
coverage estimates the probability that a new sequence will have already
been seen, OTU coverage estimates the probability that a new OTU will
match an existing one. It is therefore an extension of Good's coverage
in that it quantifies the probability that, for any given set of
sequences clustered into an OTU, that OTU will have already been seen.
Thus, high Good's coverage means that any new sequence is unlikely to be
novel, and high OTU coverage means that any new OTU is unlikely to be
novel.

\textbf{\emph{Data analysis.}} Our analysis made use of ARB (OS X v.6.0)
(43), mothur (v.1.37.0) (44), and R (v.3.2.0) (47). Within R we utilized
the knitr (v.1.10.5), wesanderson (v.0.3.2), and openxlsx (v. 2.4.0)
packages. A reproducible version of this manuscript including data
extraction and processing is available at
\url{https://www.github.com/SchlossLab/Schloss_Census2_mBio_2016}.

\newpage

\doublespacing

\textbf{Figure 1. Number of OTUs sampled among bacterial and archaeal
16S rRNA gene sequences for different OTU definitions and level of
sequencing effort.} Rarefaction curves for different OTU definitions of
Bacteria (A) and Archaea (B). Rarefaction curves for the coarse
environments in Table 1 for Bacteria (C) and Archaea (D).

\textbf{Figure 2. Progression of the archaeal and bacterial census since
the first full-length 16S rRNA gene sequence was deposited into GenBank
in 1983.}* The number of bacterial and archaeal 16S rRNA gene sequences
deposited (A) and the new OTUs they represent (B) has increased
exponentially until the last several years when the rate of change has
plateaued. For both bacterial and archaeal sequences, the number of
studies that are responsible for depositing more than 50\% of the
sequences each year has been relatively small (C).

\textbf{Figure 3. Relative rate of sequence deposition for each
bacterial and archaeal phylum before and after 2006 relative to the
sequencing of all bacteria.} The figure shows the relative rates for
those phyla with at least 1,000 sequences and the x-axis is on a log2
scale. The data for all bacterial and archaeal phyla are available in
Supplemental Tables 2 and 3, respectively.

\textbf{Figure 4. Heatmap depicting the relative abundance of the most
common bacterial and archaeal phyla across different environments.} Each
environmental category exhibited a phylum-level signature although the
bacterial census was dominated by sequences from the Firmicutes,
Proteobacteria, Actinobacteria, and Bacteroidetes and the archaeal
census was dominated by sequences from the Euryarchaeota and
Thaumarchaeota. The ten most abundant phyla across all environmental
categories are shown. The data for all bacterial and archaeal phyla are
available in Supplemental Tables 4 and 5, respectively.

\textbf{Figure 5. The rate that sequences and OTUs are generated from
bacterial and archaeal cultures relative to all sequences and OTUs by
phlum.} Phyla with greater than 1,000 sequences are listed by domain.
Open circles indicate the percentage of sequences in the database that
match cultured organisms. Closed circles indicate the percentage of OTUs
in this analysis that contain sequences belonging to a cultured
organism. The data for all bacterial and archaeal phyla are available in
Supplemental Tables 6 and 7, respectively.

\textbf{Figure 6. The percentage of bacterial and archaeal OTUs found by
single cell genomics and EMIRGE using PCR or metagenomics that were also
detected by other methods.} The bars comparing a method to itself
indicate the percentage of OTUs that were only detected by that method.

\newpage

\textbf{Supplemental Table 1. Description of environmental categories
and the criteria used to assign sequences to each category.}

\textbf{Supplementary Table 2. Frequency that each bacterial phylum was
sequenced before and after 2006.}

\textbf{Supplementary Table 3. Frequency that each archaeal phylum was
sequenced before and after 2006.}

\textbf{Supplementary Table 4. Frequency that each bacterial phylum was
found across each of the environmental categories.}

\textbf{Supplementary Table 5. Frequency that each archaeal phylum was
found across each of the environmental categories.}

\textbf{Supplementary Table 6. Frequency that each bacterial sequence or
OTU was retrieved by cultivation or by culture-independent methods.}

\textbf{Supplementary Table 7. Frequency that each archaeal sequence or
OTU was retrieved by cultivation or by culture-independent methods.}

\newpage

\hyperdef{}{references}{\label{references}}
\subsection*{References}\label{references}
\addcontentsline{toc}{subsection}{References}

\hyperdef{}{ref-McGill2007}{\label{ref-McGill2007}}
1. \textbf{McGill BJ}, \textbf{Etienne RS}, \textbf{Gray JS},
\textbf{Alonso D}, \textbf{Anderson MJ}, \textbf{Benecha HK},
\textbf{Dornelas M}, \textbf{Enquist BJ}, \textbf{Green JL}, \textbf{He
F}, \textbf{Hurlbert AH}, \textbf{Magurran AE}, \textbf{Marquet PA},
\textbf{Maurer BA}, \textbf{Ostling A}, \textbf{Soykan CU},
\textbf{Ugland KI}, \textbf{White EP}. 2007. Species abundance
distributions: Moving beyond single prediction theories to integration
within an ecological framework. Ecology Letters \textbf{10}:995--1015.
doi:\url{http://doi.org/10.1111/j.1461-0248.2007.01094.x}.

\hyperdef{}{ref-Hubbell2001}{\label{ref-Hubbell2001}}
2. \textbf{Hubbell SP}. 2001. A Unified Theory of Biodiversity and
Biogeography. Princeton University Press, Princeton.

\hyperdef{}{ref-Konstantinidis2006}{\label{ref-Konstantinidis2006}}
3. \textbf{Konstantinidis KT}, \textbf{Ramette A}, \textbf{Tiedje JM}.
2006. The bacterial species definition in the genomic era. Philosophical
Transactions of the Royal Society B: Biological Sciences
\textbf{361}:1929--1940.
doi:\url{http://doi.org/10.1098/rstb.2006.1920}.

\hyperdef{}{ref-Oren2013}{\label{ref-Oren2013}}
4. \textbf{Oren A}, \textbf{Garrity GM}. 2013. Then and now: A
systematic review of the systematics of prokaryotes in the last
80~years. Antonie van Leeuwenhoek \textbf{106}:43--56.
doi:\url{http://doi.org/10.1007/s10482-013-0084-1}.

\hyperdef{}{ref-Brosius1978}{\label{ref-Brosius1978}}
5. \textbf{Brosius J}, \textbf{Palmer ML}, \textbf{Kennedy PJ},
\textbf{Noller HF}. 1978. Complete nucleotide sequence of a 16S
ribosomal RNA gene from \emph{Escherichia coli.} Proceedings of the
National Academy of Sciences \textbf{75}:4801--4805.
doi:\url{http://doi.org/10.1073/pnas.75.10.4801}.

\hyperdef{}{ref-Schloss2004}{\label{ref-Schloss2004}}
6. \textbf{Schloss PD}, \textbf{Handelsman J}. 2004. Status of the
microbial census. Microbiology and Molecular Biology Reviews
\textbf{68}:686--691.
doi:\url{http://doi.org/10.1128/mmbr.68.4.686-691.2004}.

\hyperdef{}{ref-Dykhuizen1998}{\label{ref-Dykhuizen1998}}
7. \textbf{Dykhuizen DE}. 1998. Santa Rosalia revisited: Why are there
so many species of bacteria? Antonie van Leeuwenhoek \textbf{73}:25--33.
doi:\url{http://doi.org/10.1023/a:1000665216662}.

\hyperdef{}{ref-Curtis2002}{\label{ref-Curtis2002}}
8. \textbf{Curtis TP}, \textbf{Sloan WT}, \textbf{Scannell JW}. 2002.
Estimating prokaryotic diversity and its limits. Proceedings of the
National Academy of Sciences \textbf{99}:10494--10499.
doi:\url{http://doi.org/10.1073/pnas.142680199}.

\hyperdef{}{ref-HMP2012}{\label{ref-HMP2012}}
9. \textbf{The Human Microbiome Consortium}. 2012. Structure, function
and diversity of the healthy human microbiome. Nature
\textbf{486}:207--214. doi:\url{http://doi.org/10.1038/nature11234}.

\hyperdef{}{ref-Gilbert2014}{\label{ref-Gilbert2014}}
10. \textbf{Gilbert JA}, \textbf{Jansson JK}, \textbf{Knight R}. 2014.
The Earth Microbiome Project: Successes and aspirations. BMC Biology
\textbf{12}:69. doi:\url{http://doi.org/10.1186/s12915-014-0069-1}.

\hyperdef{}{ref-AmaralZettler2010}{\label{ref-AmaralZettler2010}}
11. \textbf{Amaral-Zettler L}, \textbf{Artigas LF}, \textbf{Baross J},
\textbf{P.A. LB}, \textbf{Boetius A}, \textbf{Chandramohan D},
\textbf{Herndl G}, \textbf{Kogure K}, \textbf{Neal P},
\textbf{Pedrós-Alió C}, \textbf{Ramette A}, \textbf{Schouten S},
\textbf{Stal L}, \textbf{Thessen A}, \textbf{Leeuw J de}, \textbf{Sogin
M}. 2010. A global census of marine microbes, pp. 221--245. \emph{In}
Life in the worlds oceans. Wiley-Blackwell.

\hyperdef{}{ref-Youssef2009}{\label{ref-Youssef2009}}
12. \textbf{Youssef N}, \textbf{Sheik CS}, \textbf{Krumholz LR},
\textbf{Najar FZ}, \textbf{Roe BA}, \textbf{Elshahed MS}. 2009.
Comparison of species richness estimates obtained using nearly complete
fragments and simulated pyrosequencing-generated fragments in 16S rRNA
gene-based environmental surveys. Applied and Environmental Microbiology
\textbf{75}:5227--5236. doi:\url{http://doi.org/10.1128/aem.00592-09}.

\hyperdef{}{ref-Schloss2010}{\label{ref-Schloss2010}}
13. \textbf{Schloss PD}. 2010. The effects of alignment quality,
distance calculation method, sequence filtering, and region on the
analysis of 16S rRNA gene-based studies. PLoS Comput Biol
\textbf{6}:e1000844.
doi:\url{http://doi.org/10.1371/journal.pcbi.1000844}.

\hyperdef{}{ref-Alivisatos2015}{\label{ref-Alivisatos2015}}
14. \textbf{Alivisatos AP}, \textbf{Blaser MJ}, \textbf{Brodie EL},
\textbf{Chun M}, \textbf{Dangl JL}, \textbf{Donohue TJ},
\textbf{Dorrestein PC}, \textbf{Gilbert JA}, \textbf{Green JL},
\textbf{Jansson JK}, \textbf{Knight R}, \textbf{Maxon ME},
\textbf{McFall-Ngai MJ}, \textbf{Miller JF}, \textbf{Pollard KS},
\textbf{Ruby EG}, \textbf{Taha SA}. 2015. A unified initiative to
harness Earth's microbiomes. Science \textbf{350}:507--508.
doi:\url{http://doi.org/10.1126/science.aac8480}.

\hyperdef{}{ref-Liux5f2012}{\label{ref-Liux5f2012}}
15. \textbf{Li E}, \textbf{Hamm CM}, \textbf{Gulati AS}, \textbf{Sartor
RB}, \textbf{Chen H}, \textbf{Wu X}, \textbf{Zhang T}, \textbf{Rohlf
FJ}, \textbf{Zhu W}, \textbf{Gu C}, \textbf{Robertson CE}, \textbf{Pace
NR}, \textbf{Boedeker EC}, \textbf{Harpaz N}, \textbf{Yuan J},
\textbf{Weinstock GM}, \textbf{Sodergren E}, \textbf{Frank DN}. 2012.
Inflammatory bowel diseases phenotype, textitC. difficile and NOD2
genotype are associated with shifts in human ileum associated microbial
composition. PLoS ONE \textbf{7}:e26284.
doi:\url{http://doi.org/10.1371/journal.pone.0026284}.

\hyperdef{}{ref-Kongux5f2012}{\label{ref-Kongux5f2012}}
16. \textbf{Kong HH}, \textbf{Oh J}, \textbf{Deming C}, \textbf{Conlan
S}, \textbf{Grice EA}, \textbf{Beatson MA}, \textbf{Nomicos E},
\textbf{Polley EC}, \textbf{Komarow HD}, \textbf{Murray PR},
\textbf{Turner ML}, \textbf{Segre JA}. 2012. Temporal shifts in the skin
microbiome associated with disease flares and treatment in children with
atopic dermatitis. Genome Research \textbf{22}:850--859.
doi:\url{http://doi.org/10.1101/gr.131029.111}.

\hyperdef{}{ref-Griceux5f2009}{\label{ref-Griceux5f2009}}
17. \textbf{Grice EA}, \textbf{Kong HH}, \textbf{Conlan S},
\textbf{Deming CB}, \textbf{Davis J}, \textbf{Young AC},
\textbf{Bouffard GG}, \textbf{Blakesley RW}, \textbf{Murray PR},
\textbf{Green ED}, \textbf{Turner ML}, \textbf{Segre JA}. 2009.
Topographical and temporal diversity of the human skin microbiome.
Science \textbf{324}:1190--1192.
doi:\url{http://doi.org/10.1126/science.1171700}.

\hyperdef{}{ref-Griceux5f2010}{\label{ref-Griceux5f2010}}
18. \textbf{Grice EA}, \textbf{Snitkin ES}, \textbf{Yockey LJ},
\textbf{Bermudez DM}, \textbf{Liechty KW}, \textbf{Segre JA},
\textbf{Mullikin J}, \textbf{Blakesley R}, \textbf{Young A}, \textbf{Chu
G}, \textbf{Ramsahoye C}, \textbf{Lovett S}, \textbf{Han J},
\textbf{Legaspi R}, \textbf{Fuksenko T}, \textbf{Reddix-Dugue N},
\textbf{Sison C}, \textbf{Gregory M}, \textbf{Montemayor C},
\textbf{Gestole M}, \textbf{Hargrove A}, \textbf{Johnson T},
\textbf{Myrick J}, \textbf{Riebow N}, \textbf{Schmidt B},
\textbf{Novotny B}, \textbf{Gupti J}, \textbf{Benjamin B},
\textbf{Brooks S}, \textbf{Coleman H}, \textbf{Ho S-l},
\textbf{Schandler K}, \textbf{Smith L}, \textbf{Stantripop M},
\textbf{Maduro Q}, \textbf{Bouffard G}, \textbf{Dekhtyar M},
\textbf{Guan X}, \textbf{Masiello C}, \textbf{Maskeri B},
\textbf{McDowell J}, \textbf{Park M}, \textbf{Thomas PJ}. 2010.
Longitudinal shift in diabetic wound microbiota correlates with
prolonged skin defense response. Proceedings of the National Academy of
Sciences \textbf{107}:14799--14804.
doi:\url{http://doi.org/10.1073/pnas.1004204107}.

\hyperdef{}{ref-Kirkux5fHarrisux5f2012}{\label{ref-Kirkux5fHarrisux5f2012}}
19. \textbf{Harris JK}, \textbf{Caporaso JG}, \textbf{Walker JJ},
\textbf{Spear JR}, \textbf{Gold NJ}, \textbf{Robertson CE},
\textbf{Hugenholtz P}, \textbf{Goodrich J}, \textbf{McDonald D},
\textbf{Knights D}, \textbf{Marshall P}, \textbf{Tufo H}, \textbf{Knight
R}, \textbf{Pace NR}. 2012. Phylogenetic stratigraphy in the guerrero
negro hypersaline microbial mat. The ISME Journal \textbf{7}:50--60.
doi:\url{http://doi.org/10.1038/ismej.2012.79}.

\hyperdef{}{ref-Sogin2006}{\label{ref-Sogin2006}}
20. \textbf{Sogin ML}, \textbf{Morrison HG}, \textbf{Huber JA},
\textbf{Welch DM}, \textbf{Huse SM}, \textbf{Neal PR}, \textbf{Arrieta
JM}, \textbf{Herndl GJ}. 2006. Microbial diversity in the deep sea and
the underexplored ``rare biosphere''. Proceedings of the National
Academy of Sciences \textbf{103}:12115--12120.
doi:\url{http://doi.org/10.1073/pnas.0605127103}.

\hyperdef{}{ref-Whitman1998}{\label{ref-Whitman1998}}
21. \textbf{Whitman WB}, \textbf{Coleman DC}, \textbf{Wiebe WJ}. 1998.
Prokaryotes: The unseen majority. Proceedings of the National Academy of
Sciences \textbf{95}:6578--6583.
doi:\url{http://doi.org/10.1073/pnas.95.12.6578}.

\hyperdef{}{ref-Rinke2013}{\label{ref-Rinke2013}}
22. \textbf{Rinke C}, \textbf{Schwientek P}, \textbf{Sczyrba A},
\textbf{Ivanova NN}, \textbf{Anderson IJ}, \textbf{Cheng J-F},
\textbf{Darling A}, \textbf{Malfatti S}, \textbf{Swan BK}, \textbf{Gies
EA}, \textbf{Dodsworth JA}, \textbf{Hedlund BP}, \textbf{Tsiamis G},
\textbf{Sievert SM}, \textbf{Liu W-T}, \textbf{Eisen JA}, \textbf{Hallam
SJ}, \textbf{Kyrpides NC}, \textbf{Stepanauskas R}, \textbf{Rubin EM},
\textbf{Hugenholtz P}, \textbf{Woyke T}. 2013. Insights into the
phylogeny and coding potential of microbial dark matter. Nature
\textbf{499}:431--437. doi:\url{http://doi.org/10.1038/nature12352}.

\hyperdef{}{ref-Miller2011}{\label{ref-Miller2011}}
23. \textbf{Miller CS}, \textbf{Baker BJ}, \textbf{Thomas BC},
\textbf{Singer SW}, \textbf{Banfield JF}. 2011. EMIRGE: Reconstruction
of full-length ribosomal genes from microbial community short read
sequencing data. Genome Biol \textbf{12}:R44.
doi:\url{http://doi.org/10.1186/gb-2011-12-5-r44}.

\hyperdef{}{ref-Miller2013}{\label{ref-Miller2013}}
24. \textbf{Miller CS}, \textbf{Handley KM}, \textbf{Wrighton KC},
\textbf{Frischkorn KR}, \textbf{Thomas BC}, \textbf{Banfield JF}. 2013.
Short-read assembly of full-length 16S amplicons reveals bacterial
diversity in subsurface sediments. PLoS ONE \textbf{8}:e56018.
doi:\url{http://doi.org/10.1371/journal.pone.0056018}.

\hyperdef{}{ref-Markowitz2013}{\label{ref-Markowitz2013}}
25. \textbf{Markowitz VM}, \textbf{Chen I-MA}, \textbf{Palaniappan K},
\textbf{Chu K}, \textbf{Szeto E}, \textbf{Pillay M}, \textbf{Ratner A},
\textbf{Huang J}, \textbf{Woyke T}, \textbf{Huntemann M},
\textbf{Anderson I}, \textbf{Billis K}, \textbf{Varghese N},
\textbf{Mavromatis K}, \textbf{Pati A}, \textbf{Ivanova NN},
\textbf{Kyrpides NC}. 2013. IMG 4 version of the integrated microbial
genomes comparative analysis system. Nucleic Acids Research
\textbf{42}:D560--D567. doi:\url{http://doi.org/10.1093/nar/gkt963}.

\hyperdef{}{ref-Wrighton2012}{\label{ref-Wrighton2012}}
26. \textbf{Wrighton KC}, \textbf{Thomas BC}, \textbf{Sharon I},
\textbf{Miller CS}, \textbf{Castelle CJ}, \textbf{VerBerkmoes NC},
\textbf{Wilkins MJ}, \textbf{Hettich RL}, \textbf{Lipton MS},
\textbf{Williams KH}, \textbf{Long PE}, \textbf{Banfield JF}. 2012.
Fermentation, hydrogen, and sulfur metabolism in multiple uncultivated
bacterial phyla. Science \textbf{337}:1661--1665.
doi:\url{http://doi.org/10.1126/science.1224041}.

\hyperdef{}{ref-NunesdaRocha2015}{\label{ref-NunesdaRocha2015}}
27. \textbf{Rocha UN da}, \textbf{Cadillo-Quiroz H}, \textbf{Karaoz U},
\textbf{Rajeev L}, \textbf{Klitgord N}, \textbf{Dunn S}, \textbf{Truong
V}, \textbf{Buenrostro M}, \textbf{Bowen BP}, \textbf{Garcia-Pichel F},
\textbf{Mukhopadhyay A}, \textbf{Northen TR}, \textbf{Brodie EL}. 2015.
Isolation of a significant fraction of non-phototroph diversity from a
desert biological soil crust. Front Microbiol \textbf{6}:277.
doi:\url{http://doi.org/10.3389/fmicb.2015.00277}.

\hyperdef{}{ref-Hamilton2015}{\label{ref-Hamilton2015}}
28. \textbf{Hamilton TL}, \textbf{Jones DS}, \textbf{Schaperdoth I},
\textbf{Macalady JL}. 2015. Metagenomic insights into S(0) precipitation
in a terrestrial subsurface lithoautotrophic ecosystem. Front Microbiol
\textbf{5}:756. doi:\url{http://doi.org/10.3389/fmicb.2014.00756}.

\hyperdef{}{ref-Handley2012}{\label{ref-Handley2012}}
29. \textbf{Handley KM}, \textbf{VerBerkmoes NC}, \textbf{Steefel CI},
\textbf{Williams KH}, \textbf{Sharon I}, \textbf{Miller CS},
\textbf{Frischkorn KR}, \textbf{Chourey K}, \textbf{Thomas BC},
\textbf{Shah MB}, \textbf{Long PE}, \textbf{Hettich RL},
\textbf{Banfield JF}. 2012. Biostimulation induces syntrophic
interactions that impact c, s and n cycling in a sediment microbial
community. The ISME Journal \textbf{7}:800--816.
doi:\url{http://doi.org/10.1038/ismej.2012.148}.

\hyperdef{}{ref-Gladden2011}{\label{ref-Gladden2011}}
30. \textbf{Gladden JM}, \textbf{Allgaier M}, \textbf{Miller CS},
\textbf{Hazen TC}, \textbf{VanderGheynst JS}, \textbf{Hugenholtz P},
\textbf{Simmons BA}, \textbf{Singer SW}. 2011. Glycoside hydrolase
activities of thermophilic bacterial consortia adapted to switchgrass.
Applied and Environmental Microbiology \textbf{77}:5804--5812.
doi:\url{http://doi.org/10.1128/aem.00032-11}.

\hyperdef{}{ref-Brooks2014}{\label{ref-Brooks2014}}
31. \textbf{Brooks B}, \textbf{Firek BA}, \textbf{Miller CS},
\textbf{Sharon I}, \textbf{Thomas BC}, \textbf{Baker R},
\textbf{Morowitz MJ}, \textbf{Banfield JF}. 2014. Microbes in the
neonatal intensive care unit resemble those found in the gut of
premature infants. Microbiome \textbf{2}:1.
doi:\url{http://doi.org/10.1186/2049-2618-2-1}.

\hyperdef{}{ref-Wilkins2013}{\label{ref-Wilkins2013}}
32. \textbf{Wilkins MJ}, \textbf{Wrighton KC}, \textbf{Nicora CD},
\textbf{Williams KH}, \textbf{McCue LA}, \textbf{Handley KM},
\textbf{Miller CS}, \textbf{Giloteaux L}, \textbf{Montgomery AP},
\textbf{Lovley DR}, \textbf{Banfield JF}, \textbf{Long PE},
\textbf{Lipton MS}. 2013. Fluctuations in species-level protein
expression occur during element and nutrient cycling in the subsurface.
PLoS ONE \textbf{8}:e57819.
doi:\url{http://doi.org/10.1371/journal.pone.0057819}.

\hyperdef{}{ref-Handley2014}{\label{ref-Handley2014}}
33. \textbf{Handley KM}, \textbf{Wrighton KC}, \textbf{Miller CS},
\textbf{Wilkins MJ}, \textbf{Kantor RS}, \textbf{Thomas BC},
\textbf{Williams KH}, \textbf{Gilbert JA}, \textbf{Long PE},
\textbf{Banfield JF}. 2014. Disturbed subsurface microbial communities
follow equivalent trajectories despite different structural starting
points. Environ Microbiol \textbf{17}:622--636.
doi:\url{http://doi.org/10.1111/1462-2920.12467}.

\hyperdef{}{ref-Alessi2014}{\label{ref-Alessi2014}}
34. \textbf{Alessi DS}, \textbf{Lezama-Pacheco JS}, \textbf{Janot N},
\textbf{Suvorova EI}, \textbf{Cerrato JM}, \textbf{Giammar DE},
\textbf{Davis JA}, \textbf{Fox PM}, \textbf{Williams KH}, \textbf{Long
PE}, \textbf{Handley KM}, \textbf{Bernier-Latmani R}, \textbf{Bargar
JR}. 2014. Speciation and reactivity of uranium products formed during
in situ bioremediation in a shallow alluvial aquifer. Environmental
Science \& Technology \textbf{48}:12842--12850.
doi:\url{http://doi.org/10.1021/es502701u}.

\hyperdef{}{ref-Locey2015}{\label{ref-Locey2015}}
35. \textbf{Locey KJ}, \textbf{Lennon JT}. 2015. Scaling laws predict
global microbial diversity. PeerJ PrePrints.
doi:\url{http://doi.org/10.7287/peerj.preprints.1451v1}.

\hyperdef{}{ref-Parada2015}{\label{ref-Parada2015}}
36. \textbf{Parada AE}, \textbf{Needham DM}, \textbf{Fuhrman JA}. 2015.
Every base matters: Assessing small subunit rRNA primers for marine
microbiomes with mock communities, time series and global field samples.
Environ Microbiol n/a--n/a.
doi:\url{http://doi.org/10.1111/1462-2920.13023}.

\hyperdef{}{ref-Brown2015}{\label{ref-Brown2015}}
37. \textbf{Brown CT}, \textbf{Hug LA}, \textbf{Thomas BC},
\textbf{Sharon I}, \textbf{Castelle CJ}, \textbf{Singh A},
\textbf{Wilkins MJ}, \textbf{Wrighton KC}, \textbf{Williams KH},
\textbf{Banfield JF}. 2015. Unusual biology across a group comprising
more than 15\% of domain bacteria. Nature \textbf{523}:208--211.
doi:\url{http://doi.org/10.1038/nature14486}.

\hyperdef{}{ref-EloeFadrosh2016}{\label{ref-EloeFadrosh2016}}
38. \textbf{Eloe-Fadrosh EA}, \textbf{Ivanova NN}, \textbf{Woyke T},
\textbf{Kyrpides NC}. 2016. Metagenomics uncovers gaps in amplicon-based
detection of microbial diversity. Nat Microbiol 15032.
doi:\url{http://doi.org/10.1038/nmicrobiol.2015.32}.

\hyperdef{}{ref-Schloss2015}{\label{ref-Schloss2015}}
39. \textbf{Schloss PD}, \textbf{Westcott SL}, \textbf{Jenior ML},
\textbf{Highlander SK}. 2015. Sequencing 16S rRNA gene fragments using
the pacBio sMRT dNA sequencing system. PeerJ PrePrints.
doi:\url{http://doi.org/10.7287/peerj.preprints.778v1}.

\hyperdef{}{ref-Nichols2010}{\label{ref-Nichols2010}}
40. \textbf{Nichols D}, \textbf{Cahoon N}, \textbf{Trakhtenberg EM},
\textbf{Pham L}, \textbf{Mehta A}, \textbf{Belanger A}, \textbf{Kanigan
T}, \textbf{Lewis K}, \textbf{Epstein SS}. 2010. Use of iChip for
high-throughput in situ cultivation of ``uncultivable'' microbial
species. Applied and Environmental Microbiology \textbf{76}:2445--2450.
doi:\url{http://doi.org/10.1128/aem.01754-09}.

\hyperdef{}{ref-Buerger2012}{\label{ref-Buerger2012}}
41. \textbf{Buerger S}, \textbf{Spoering A}, \textbf{Gavrish E},
\textbf{Leslin C}, \textbf{Ling L}, \textbf{Epstein SS}. 2012. Microbial
scout hypothesis, stochastic exit from dormancy, and the nature of slow
growers. Applied and Environmental Microbiology \textbf{78}:3221--3228.
doi:\url{http://doi.org/10.1128/aem.07307-11}.

\hyperdef{}{ref-Das2015}{\label{ref-Das2015}}
42. \textbf{Das N}, \textbf{Tripathi N}, \textbf{Basu S}, \textbf{Bose
C}, \textbf{Maitra S}, \textbf{Khurana S}. 2015. Progress in the
development of gelling agents for improved culturability of
microorganisms. Front Microbiol \textbf{6}:698.
doi:\url{http://doi.org/10.3389/fmicb.2015.00698}.

\hyperdef{}{ref-Pruesse2007}{\label{ref-Pruesse2007}}
43. \textbf{Pruesse E}, \textbf{Quast C}, \textbf{Knittel K},
\textbf{Fuchs BM}, \textbf{Ludwig W}, \textbf{Peplies J},
\textbf{Glockner FO}. 2007. SILVA: A comprehensive online resource for
quality checked and aligned ribosomal RNA sequence data compatible with
ARB. Nucleic Acids Research \textbf{35}:7188--7196.
doi:\url{http://doi.org/10.1093/nar/gkm864}.

\hyperdef{}{ref-Schloss2009}{\label{ref-Schloss2009}}
44. \textbf{Schloss PD}, \textbf{Westcott SL}, \textbf{Ryabin T},
\textbf{Hall JR}, \textbf{Hartmann M}, \textbf{Hollister EB},
\textbf{Lesniewski RA}, \textbf{Oakley BB}, \textbf{Parks DH},
\textbf{Robinson CJ}, \textbf{Sahl JW}, \textbf{Stres B},
\textbf{Thallinger GG}, \textbf{Horn DJV}, \textbf{Weber CF}. 2009.
Introducing mothur: Open-source, platform-independent,
community-supported software for describing and comparing microbial
communities. Applied and Environmental Microbiology
\textbf{75}:7537--7541. doi:\url{http://doi.org/10.1128/aem.01541-09}.

\hyperdef{}{ref-Westcott2015}{\label{ref-Westcott2015}}
45. \textbf{Westcott SL}, \textbf{Schloss PD}. 2015. De novo clustering
methods outperform reference-based methods for assigning 16S rRNA gene
sequences to operational taxonomic units. PeerJ \textbf{3}:e1487.
doi:\url{http://doi.org/10.7717/peerj.1487}.

\hyperdef{}{ref-Good1953}{\label{ref-Good1953}}
46. \textbf{Good IJ}. 1953. The population frequencies of species and
the estimation of population parameters. Biometrika
\textbf{40}:237--264.
doi:\url{http://doi.org/10.1093/biomet/40.3-4.237}.

\hyperdef{}{ref-language2015}{\label{ref-language2015}}
47. \textbf{R Core Team}. 2015. R: A language and environment for
statistical computing. R Foundation for Statistical Computing, Vienna,
Austria.

\end{document}
